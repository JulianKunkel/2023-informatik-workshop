%%%%%%%%%%%%%%%%%%%%%%%%%%%%%%%%%%%%%%%%%
% Exercise Template
% LaTeX Template
% Version 1.1 (2023-06-30)
%
% This template was adapted by:
% Jonathan Decker (jonathan.decker@uni-goettingen.de)
% From a template made by:
% Julian Kunkel (julian.kunkel@gwdg.de)
%
%%%%%%%%%%%%%%%%%%%%%%%%%%%%%%%%%%%%%%%%%
% This file provides the generic header that sets up the page formatting and
% common custom macro and environment definitions that can be used for an exercise or tutorial sheet in a university course.
%
% You can consult the `exercise.sty` file for a full list of available macros.

\documentclass[paper=a4]{scrartcl}

\newcommand{\theauthor}{Lars Quentin}
\newcommand{\thecourse}{Introduction to Git} % Course long name
\newcommand{\thecourseabbr}{AK-OSS} % Course abbreviation
\newcommand{\thetype}{Exercise} % e.g. Exercise, Tutorial
\newcommand{\theorganization}{INFORMATIK 2023} % e.g. University Göttingen
\newcommand{\theinstitution}{Arbeitskreis Open Source Software} % e.g. Institute of Computer Science
\newcommand{\theyear}{} % The year
\newcommand{\theexercisenumber}{1} % number of the exercise
\newcommand{\thedate}{27.09.2023} % e.g. March 28, 2022
\newcommand{\theterm}{AK-OSS} % SoSe or WiSe

\usepackage{exercise} % Custom header and macros

\begin{document}
\date{\thedate}
\exercise{\theexercisenumber}

\parskip 8pt
\makesheetheader

\section*{Exercise Introduction}
What to expect and what this sheet will cover.

\tableofcontents

\bigskip

\task{Set up your Environment}{5}

Ensure that you have the proper environment for your future work!

\subsection*{Set up your Git environment}

First, you need to install git. The instructions can be found at \href{https://git-scm.com/downloads}{\url{https://git-scm.com/downloads}}\\

Next, install the GitHub client.
\begin{itemize}
  \item For Windows and macOS: \href{https://desktop.github.com/}{\url{https://desktop.github.com/}}
  \item For Linux (community maintained): \href{https://github.com/shiftkey/desktop}{\url{https://github.com/shiftkey/desktop}}
\end{itemize}

Next, you have to configure your git. In a terminal, type the following:
\begin{enumerate}
	\item \cmd{git config --global user.name "Your name"}
  \item \cmd{git config --global user.email "youremail@example.com"}
\end{enumerate}

For command line use, you'd also have to \href{https://docs.github.com/en/authentication/connecting-to-github-with-ssh}{setup an SSH key}, but its not needed for the GitHub client.\\

Lastly, log into your GitHub client.

\subsection*{Optional: Install a proper text editor}

Ensure that you have a proper editor for this and the following exercises. We recommend:

\newpage

\begin{itemize}
  \item For beginners:
    \begin{itemize}
      \item Windows: \href{https://notepad-plus-plus.org/}{Notepad++}
      \item Linux: \href{https://wiki.gnome.org/Apps/Gedit}{gedit} (GNOME) oder \href{https://kate-editor.org/}{kate} (KDE)
      \item macOS: TextEdit (pre-installed)
    \end{itemize}
  \item For advanced users:
    \begin{itemize}
      \item \href{https://code.visualstudio.com/}{Visual Studio Code}
      \item Or whichever editor you already use...
    \end{itemize}
\end{itemize}

\task{Create and Clone a GitHub repository}{5}

\subsection*{Create a local GitHub repository}

Here are the steps:
\begin{itemize}
  \item Choose a folder with a few files, preferably text-files
    \begin{itemize}
      \item Or just create some random text files.
    \end{itemize}
  \item Initialize the repository using the GitHub client
    \begin{itemize}
      \item \texttt{FILE -> New Repository}
    \end{itemize}
  \item Add all files as changes; commit them.
  \item Publish repository using the \texttt{Publish Repository} button
\end{itemize}

\subsection*{Clone a remote GitHub repository}


\begin{itemize}
  \item Clone the following repo using the GitHub client: \href{https://github.com/github/gitignore}{\url{https://github.com/github/gitignore}}
  \item Play around, change a file and see the difference in the GitHub client!
  \item Write a commit message. Do the local commit.
\end{itemize}

\begin{hints}
  \item Since you do not have access, you cannot push your local changes
\end{hints}

\end{document}

%%%%%%%%%%%%%%%%%%%%%%%%%%%%%%%%%%%%%%%%%
% Presentation Template
% LaTeX Template
% Version 2.2 (2023-07-17)
%
% This template was adapted by:
% Jonathan Decker (jonathan.decker@uni-goettingen.de)
% From a template made by:
% Julian Kunkel (julian.kunkel@gwdg.de)
%
%%%%%%%%%%%%%%%%%%%%%%%%%%%%%%%%%%%%%%%%%
\documentclass[compress,aspectratio=169]{beamer}

% make sure the theme and config files are on this path
\usepackage[GI]{assets/beamerConfig}

\addbibresource{ref.bib}
\graphicspath{{.}{assets/}}

% --- document configuration ---
\newcommand{\mytitle}{Workflow für das Beitragen von Code zu bestehenden Projekten}
% Leave empty for no subtitle
\newcommand{\mysubtitle}{Am Beispiel von GitHub}
\newcommand{\myauthor}{Nico Schwind}
\newcommand{\myauthorurl}{}
\newcommand{\myvenue}{Informatik 2023}
% For example, use \today
\newcommand{\mydate}{\today}
% For example, Institute for Computer Science / GWDG / Uni Göttingen
\newcommand{\myinstitute}{Universität Würzburg}

\configuretitlepage


\begin{document}

\begin{frame}[plain]
	\titlepage
\end{frame}

\begin{frame}[t]{Table of contents}
  \tableofcontents[subsectionstyle=hide/hide]
\end{frame}

% --- slides begin ---

\section{Grundlagen}

\begin{frame}{Möglichkeiten finden, um zu Open-Source Projekten beizutragen}
  \begin{itemize}
    \item Nutze github.com/topics/<topic> um nach Themen zu suchen, die dich interessieren
    \item 
  \end{itemize}
\end{frame}

\begin{frame}{Itemize within itemize}
    \begin{itemize}
        \item List
            \begin{itemize}
                \item Sub-bullet
            \end{itemize}
    \end{itemize}
\end{frame}

\begin{frame}{Block and negative space}
    \vspace*{-3em} % Make more space by moving everything up
    \begin{block}{Block}
        \begin{itemize}
            \item
        \end{itemize}
    \end{block}
\end{frame}


\section{Section 2}
\sectionIntroHidden % Show an outline of the current section with hidden subsections
%\sectionIntro % Show an outline of the current section with subsections

\begin{frame}{Includegraphics}
    \centering
    \includegraphics[width=0.6\textwidth]{assets/GI_Logo_AK-OSS.png}\\
    \source{Image source: \url{https://ak-oss.gi.de/}}
\end{frame}

\subsection{Subsec 1} % Subsections are used for highlighting parts of the top navigation bar, the names only show up in the sectionIntro
\begin{frame}{Pause}
    \begin{itemize}
        \item 1
        \pause  % Create separate slides in the pdf to show points one at a time
        \item 2
        \pause
        \item 3
    \end{itemize}
\end{frame}

\begin{frame}{Columns}
    \begin{columns}
    \begin{column}{0.5\textwidth}
    Left column
    \end{column}
    \begin{column}{0.5\textwidth}
    Right column % Can also be used for inserting images next to text
    \end{column}
    \end{columns}
\end{frame}

\begin{frame}{Tikz graphics}
    % Use tikz for placing images based on page coordinates
    \begin{tikzpicture}[remember picture,overlay]
      \node [xshift=-4.8cm,yshift=-3.8cm] at (current page.north east)
        {\includegraphics[width=4cm]{assets/GI_Logo_AK-OSS.png}};
    \end{tikzpicture}
    \source{Image source: \url{https://ak-oss.gi.de/}}
\end{frame}

\begin{frame}[fragile]{Code listing}
	
	\texttt{Simple HTTP Server} implemented in Python
	
	\begin{tcolorbox}[title=Python]
		\footnotesize\begin{minted}[xleftmargin=1em,linenos]{Python}
from http.server import HTTPServer, BaseHTTPRequestHandler

class SimpleHTTPRequestHandler(BaseHTTPRequestHandler):
	def do_GET(self):
		self.send_response(200)
		self.end_headers()
		self.wfile.write(b'Hello, world!')

HOSTNAME = "localhost"
PORT = 8000
httpd = HTTPServer((HOSTNAME, PORT), SimpleHTTPRequestHandler)
print("Server starting at http://" + HOSTNAME + ":" + str(PORT))
httpd.serve_forever()
		\end{minted}
	\end{tcolorbox}
\end{frame}

\begin{frame}[fragile]{Code listing from file}
    % Use minted package for handling code listings on slides
    \texttt{Hello World} implemented in C++

    \begin{tcolorbox}[title=C++]
        \footnotesize\inputminted[xleftmargin=1em,linenos]{c++}{assets/hello-world.cpp}
    \end{tcolorbox}

\end{frame}

\begin{frame}{Todonotes}
    \todo{Set todo reminders for yourself}
\end{frame}

\begin{frame}{Quotes}
    % Make a quote the central element of a slide to emphasize its importance
    \vspace*{\fill}
    \begin{quote}
        \centering\Large
        \enquote{Non-reproducible single occurrences are of no significance to science.}
    \end{quote}
    \vspace{1cm}
    \hspace*\fill{\small--- Karl Popper, The Logic of Scientific Discovery, 2002, p.\,66}
    \vspace*{\fill}

    \source{\cite{popperLogicScientificDiscovery2002}}
\end{frame}


\section{Section 3}

\begin{frame}{Rounded Corners for Images}
	\begin{columns}
		\begin{column}{0.5\textwidth}
			\begin{itemize}
				\item Use the \texttt{\textbackslash cutpic} command to round corners of images
				\item Usage: \texttt{\textbackslash cutpic\{corner size\}\{width\}\{file\}}
			\end{itemize}
		\end{column}
		\begin{column}{0.5\textwidth}
			\vspace*{-0.2cm}
			\cutpic{0.3cm}{\textwidth}{example-image}
		\end{column}
	\end{columns}
\end{frame}

\begin{frame}[t]{Sequentially Reveal Images}
	\begin{columns}[T]
		\begin{column}{0.6\textwidth}
			\begin{itemize}
				\begin{onlyenv}<1->
					\item Sequentially reveal text
				\end{onlyenv}
				\begin{onlyenv}<2->
					\item Along with images
				\end{onlyenv}
				\begin{onlyenv}<3->
					\item Adapt this to your needs
				\end{onlyenv}
			\end{itemize}
		\end{column}
		\begin{column}{0.4\textwidth}
			\begin{tikzpicture}
				\begin{onlyenv}<1->
					\node at (0,0) (img1)	{\cutpic{0.3cm}{\textwidth}{example-image-a}};
				\end{onlyenv}
				\begin{onlyenv}<2->
					\node at (0,-0.5) (img2)	{\cutpic{0.3cm}{\textwidth}{example-image-b}};
				\end{onlyenv}
				\begin{onlyenv}<3->
					\node at (0,-1) (img3)	{\cutpic{0.3cm}{\textwidth}{example-image-c}};
				\end{onlyenv}
			\end{tikzpicture}
		\end{column}
	\end{columns}
	\source{Put Multiple sources\\
		one per line}
\end{frame}

\begin{frame}{Last Frame}
\label{pg:lastpage} % Label on last frame to get the page number for footer

\end{frame}

\begin{frame}{References}
    % References slide in appendix
    \renewcommand*{\bibfont}{\normalfont\scriptsize}
    \printbibliography[heading=none]
\end{frame}

\end{document}

% This file provides the generic header that sets up the page formatting and
% common custom macro and environment definitions that can be used for an exercise
% or tutorial sheet in a university course.
%
% You can consult the `00-Header.sty` file for a full list of available macros.
%
% --------------------80-CHARACTER-SOURCE-LINE-WIDTH-GUIDE----------------------

% The documentclass is set to an a4 paper article using KOMA-script
\documentclass[paper=a4]{scrartcl}

\newcommand{\theauthor}{Jonathan Decker}
\newcommand{\thecourse}{Practical course on High-Performance Computing}% Course long name
\newcommand{\thecourseabbr}{PCHPC} % Course abbreviation
\newcommand{\thetype}{Git Exercise} % e.g. Exercise, Tutorial
\newcommand{\theorganization}{Universität Göttingen} % e.g. University Göttingen
\newcommand{\theinstitution}{Institute for Computer Science / GWDG} % e.g. Department of Computer Science
\newcommand{\theyear}{2023} %
\newcommand{\theexercisenumber}{1} % number of the exercise
\newcommand{\thedate}{2023.04.17} % e.g. March 28, 2022
\newcommand{\theterm}{SoSe} % SoSe or WiSe 

\usepackage{00-Header} % Custom header and macros
\usepackage{longtable}
\usepackage{textcomp}

\definecolor{bgcolor}{HTML}{E0E0E0}
\let\oldtexttt\texttt

\renewcommand{\texttt}[1]{
  \colorbox{bgcolor}{\oldtexttt{#1}}
  }

\begin{document}
\date{\thedate}
\exercise{\theexercisenumber}

\parskip 8pt
\makesheetheader

\section*{Exercise Introduction}
Before attempting the exercises in this document please ensure that you have read and understood the key topics covered in Tutorial.

\tableofcontents

\bigskip

%\section*{Git}
\task{Basic Git Setup}{5}
Run the following commands and observe what they do. Feel free to test around. \\
You can find help for any git command using \texttt{git <COMMAND> -{}-help} or \texttt{man git-<COMMAND>.}\\
Feel free to replace \texttt{nano} with your favorite text editor command.

\noindent
\begin{longtable}{ p{.5\textwidth} p{.5\textwidth}}
        \textbf{Setup}\\
        \hline
        \texttt{mkdir -p \$HOME/git-exercise \&\& cd \$HOME/git-exercise}\\
        \texttt{git -{}-version}\\
        \texttt{git config --global user.name "NAME"} & Set your name.\\
        \texttt{git config --global user.email "EMAIL"} & Set your email.\\
        \texttt{git config --global core.editor "nano"} & Set nano as editor for commit messages.\\
        \texttt{git init --initial-branch=main}\\
        \texttt{git status}\\
\end{longtable}

\task{Commits and branches}{10}
\begin{longtable}{ p{.5\textwidth} p{.5\textwidth}}
          \textbf{Committing}\\
        \hline
        \texttt{touch README}\\
        \texttt{git status}\\
        \texttt{git add .}\\
        \texttt{git status}\\
        \texttt{git commit -m "Initial Commit"} \\
        \texttt{nano README} & Write a few words into the file and close nano.\\
        \texttt{git status}\\
        \texttt{git diff README} & See your changes, close with \keys{q}.\\
        \texttt{git add README}\\
        \texttt{git commit} & Write a commit message using nano, save and close.\\
         & \\

        \textbf{Reverting changes}\\
        \hline
        \texttt{rm README}\\
        \texttt{git status}\\
        \texttt{git reset --hard HEAD} & Undo the delete by reverting to the last commit\\
        & this also undoes any other changes you made.\\
        \texttt{git status}\\
        \texttt{ls} & See that the README file is back.\\
        \texttt{rm README}\\
        \texttt{git commit -a -m "Deleted README"} & Use \textbf{-a} flag to commit a staged changes.\\
        \texttt{git status}\\
        \texttt{ls} & Confirm that README was deleted and the change was committed.\\
        \texttt{git revert \keys{TAB + TAB}} & This shows a list of your recent commits.\\
        & Type the first two characters of the id of your last commit and\\
        & press \keys{TAB} and \keys{ENTER}.\\
        & Write a commit message for your reverted commit, save and close nano.\\

        \textbf{Creating branches}\\
        \hline
        \texttt{git checkout -b feature} & Create a new branch.\\
        \texttt{echo "This is a new file." > new\_file} & \\
        \texttt{git add new\_file} & Stage the new file.\\
        \texttt{git commit -m"add a new file"} & Create a commit on \textbf{feature}.\\
        \texttt{git log --graph --oneline --decorate} & Visualize the commit history.\\
        \texttt{git checkout main} & Return to the initial branch.\\
        \texttt{ls} & What happened to the working tree?\\
      \end{longtable}

      \task{Remote repositories}{5}
      \begin{longtable}{ p{.5\textwidth} p{.5\textwidth}}      
        \textbf{Remote repository}\\
        \hline
        & Make sure you can login to \url{https://gitlab.gwdg.de},\\
        & \url{https://gitlab-ce.gwdg.de} or \url{https://github.com}. \\
        & Replace the domain and username accordingly.\\
        \texttt{git remote add origin "https://gitlab-ce.gwdg.de/USERNAME/git-exercise.git}\\
        \texttt{git push --set-upstream origin main} & This will query your credentials if you do not have them stored already\\
        & and create a remote repository.\\
        & The visibility of the repository is private by default\\
        & so only you and the teammates you have explicitly invited have access.\\
        & Visit \url{https://gitlab-ce.gwdg.de/USERNAME/git-exercise} to view your new project.\\
        \texttt{rm -rf $\tilde{~}$/git-exercise} & Delete the local copy of the repository.\\
        \texttt{cd} & switch back to your home directory.\\
        \texttt{git clone https://gitlab-ce.gwdg.de/USERNAME/git-exercise.git}\\
        & Download the repository from remote.\\
        \texttt{cd git-exercise}\\
        & Make a change to README on the web and commit it.\\
        \texttt{git pull}\\
        \texttt{git log} & See the change you made on the web.\\
\end{longtable}

\task{Working with .gitignore}{2}
\begin{longtable}{ p{.5\textwidth} p{.5\textwidth}}
      
        \textbf{gitignore}\\
        \hline
        \texttt{touch credentials}\\
        \texttt{git status} & See that the credentials file can be staged.\\
        \texttt{nano .gitignore} & Write \textbf{credentials} into the file and save your changes.\\
        \texttt{git status} & See that only \textbf{.gitignore} can be staged and credentials is ignored.\\
        \texttt{git add credentials}\\
\end{longtable}

Git has many more features, one of them, which is commonly used, is branching.
\begin{literature}
  \item Missing Semester; Version Control (Git): \url{https://missing.csail.mit.edu/2020/version-control/}
  \item Learn Git branching: \url{https://learngitbranching.js.org/}
\end{literature}

\end{document}

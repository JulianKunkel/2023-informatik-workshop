\PassOptionsToPackage{hyphens}{url}
\documentclass[compress,aspectratio=169]{beamer}

%\usetheme{Goettingen} % Generates a sidebar that is replaced by a header bar in the given theme
\usepackage[official]{eurosym}
\usepackage{multirow}
\usepackage{units}
\usepackage{../Git/Slides/assets/beamerthemeGoettingen} % make sure the theme file is on this path
\usepackage{tikz}
\usetikzlibrary{shapes}
\usepackage{menukeys}
\definecolor{bgcolor}{HTML}{E0E0E0}
\newcommand{\console}[1]{
  \colorbox{bgcolor}{\texttt{#1}}
}
\graphicspath{{../}{../Git/Slides/assets/}}


% --- document configuration ---
\newcommand{\mytitle}{Workshop: Zukunft nachhaltig mit Hilfe der  \\Open Source Bewegung gestalten} 
% Leave empty for no subtitle
\newcommand{\mysubtitle}{Willkommen}
\newcommand{\myauthor}{Julian Kunkel}
\newcommand{\myauthorurl}{julian.kunkel@gwdg.de}
\newcommand{\myvenue}{}
% For example, use \today
\newcommand{\mydate}{2023.04.17}
% For example, Institute for Computer Science / GWDG
\newcommand{\myinstitute}{Institute for Computer Science / GWDG}
% Leave empty for no footer image
\newcommand{\myfooterimage}{}
\newcommand{\mygrouplogo}{hps-logo}
% Images must be enabled manually under title page \titleLogo
% Adjust position and width manually for fewer images
\newcommand{\mytitleimageone}{}
\newcommand{\mytitleimagetwo}{}
\newcommand{\mytitleimagethree}{}


% --- title page ---
\title{\Large \mytitle}
\venue{\myvenue}
\date{\mydate}
\subtitle{\mysubtitle}
\authorURL{\myauthorurl}
\author{{\myauthor}}
\authorFooter{\myauthor \hspace{0.3cm} \includegraphics[height=1em]{\myfooterimage}}
\institute{\myinstitute}
\groupLogo{\includegraphics[width=2cm]{\mygrouplogo}}
\titleLogo{
%\includegraphics[height=2.7cm]{\mytitleimageone}
%\includegraphics[height=2.7cm]{\mytitleimagetwo}
%\includegraphics[height=2.7cm]{\mytitleimagethree}
}

\setbeamertemplate{footline}[text line]{
\begin{beamercolorbox}[sep=0.5em,wd=\paperwidth,leftskip=0.2cm,rightskip=0.1cm]{footlinecolor}
\myauthor \hfill \insertVenue \hfill \insertframenumber\,/\,\inserttotalframenumber%\ref{pg:lastpage}
\end{beamercolorbox}
}

\begin{document}

\begin{frame}[plain]
	\titlepage
\end{frame}

\begin{frame}[t]{Table of contents}
  \tableofcontents[subsectionstyle=hide/hide]
\end{frame}

% --- slides begin ---

\section{Einführung}

\begin{frame}{Open Source: Kurze Definition}
  Open Source: Software, deren Quelltext 
  \begin{itemize}
    \item \textbf{öffentlich} ist und von 
    \item \textbf{Dritten eingesehen, geändert und genutzt} werden kann
  \end{itemize}
  Open-Source-Software kann unter Einhaltung der Lizenzbedingungen meistens kostenfrei genutzt werden \\
  \hfill \href{https://de.wikipedia.org/wiki/Open_Source}{Quelle: Wikipedia}
\end{frame}

\begin{frame}{Motivation für den Workshop}
  \begin{itemize}
    \item Nutzen und Potential von Open Source Software wird oft missverstanden
      \begin{itemize}
        \item Persönliche, wirtschaftliche und gesellschaftliche Relevanz
      \end{itemize}
    \item Open Source ist essentiell für digitale Unabhängigkeit 
    \begin{itemize}
      \item Ermächtigt Entwickler eigene Projekte umzusetzen 
      \item Eigene Projekte öffentlich zu präsentieren
      \item Gemeinsam kann Großes erreicht werden
    \end{itemize}
    \item Es gibt oftmals Hürden, zu bestehenden Projekten beizutragen
    \begin{itemize}
      \item Fehlendes Wissen um Prozesse und Techniken
      \item Psychologische Barrieren
    \end{itemize}
  \end{itemize}
\end{frame}

\begin{frame}{Ziele des Workshops}
  \begin{block}{Teilnehmende sollen ermächtigt werden}
  \begin{itemize}
    \item zu Open-Source Projekten beitragen zu können
    \item ein öffentliches Entwicklerprofil zeigen zu können
    \item Hintergründe für die Open Source Bewegung zu verstehen
    \item wichtigste Lizenzmodelle zu unterscheiden
  \end{itemize}
  \end{block}

  \begin{block}{Bestandteile des Workshops}
    \begin{itemize}
      \item Theorie
      \item Tutorials und Übungen
      \item Diskussion
      \item Hackathon - Eigene Beiträge/Ideen zu bestehenden Projekten
    \end{itemize}
  \end{block}
\end{frame}


\begin{frame}{Zielgruppe}
  
  \begin{block}{1. Teil: für jeden}
    \begin{itemize}
      \item Code-Entwickler
      \item Open-Source Interessierte
    \end{itemize}
    \end{block}

  \begin{block}{2. Teil: für Code-Beginner/Praktiker}
  \begin{itemize}
    \item Einstieg für Entwickler zu OSS-Beiträgen
    \item Teilnahme mit eigenem Laptop
  \end{itemize}
  \end{block}
\end{frame}

\begin{frame}{Agenda}
  \begin{itemize}
    \item Willkommen - Julian
    \item Hintergründe zur Open Source Bewegung - Sebastian
    \item Digitale Souveränität - Harald
    \item Plattformen für das Teilen von Code - Jonas
    \item Lizenzfragen - Lars \\
    \hrule
    \item Grundlagen der Versionsverwaltung mit Git - Lars
    \item Schritte zur Publizierung eigenen Codes - Marc-Philipp
    \item Workflow für das Beitragen von Code - Nico
    \item Hackathon - Alle 
    \item \textit{Alternativ: OSS and Digital Sovereignity in Asia - Mahmad Farrahi}
  \end{itemize}
\end{frame}

\section{Organisatorisches}

\sectionIntroHidden

\begin{frame}{Organisation}
  \begin{itemize}
    \item Der Workshop wurde Organisiert vom Arbeitskreis Open Source Software \\
    \url{https://ak-oss.gi.de/}
    \item Unterstützt von der Universität Göttingen und Universität Würzburg
  \end{itemize}


  \begin{block}{Involvierte Personen}
    Julian Kunkel,
    Lars Quentin,
    Jonathan Decker,
    Harald Wehnes,
    Jonas Michler,
    Marc-Philipp Knechtle,
    Nico Schwind,
    Mamad Farrahi,
    Sebastian Bergmann
  \end{block}
\end{frame}



\begin{frame}{Workshop-Mentalität}
  \begin{itemize}
    \item Knigge: Wir Duzen uns
    \item Der Workshop soll Spaß machen...
    \item Diskussion und Fragen haben Vorrang gegenüber detailliertem Inhalt
    \item Fragen werden für Online-Teilnehmende am Präsentations-Laptop wiederholt
    \item Pausen - nach Teil 1. und während praktischer Übungen im Teil 2 jederzeit
    \item Online-Teilnehmende: Fragen gerne jederzeit in den Chat
  \end{itemize}
\end{frame}

\begin{frame}{Umfrage}
  Abfrage von Hintergründen kurz mit Handzeichen (bzw. Online via "+" und "-")
  \begin{itemize}
    \item Erfahrung mit Git-Versionsverwaltung (ja/viel - melden bzw. +)
    \item Erfahrung mit Github oder anderen Code-Plattformen    
  \end{itemize}

 Was interessiert euch am meisten? (kurze Wortmeldungen)
\end{frame}


\begin{frame}{Weitere Information}
  \begin{itemize}
    \item AK OSS Webseite \url{https://ak-oss.gi.de/veranstaltung/information/zukunft-nachhaltig-mit-hilfe-der-open-source-bewegung-gestalten} 
    \item Wir organisieren typischerweise monatlich einen Event z.B. Themennachmittag/Abend
    \item Materialien des Workshops \url{https://github.com/JulianKunkel/2023-informatik-workshop}
  \end{itemize}

\end{frame}

\end{document}

%%%%%%%%%%%%%%%%%%%%%%%%%%%%%%%%%%%%%%%%%
% Presentation Template
% LaTeX Template
% Version 2.2 (2023-07-17)
%
% This template was adapted by:
% Jonathan Decker (jonathan.decker@uni-goettingen.de)
% From a template made by:
% Julian Kunkel (julian.kunkel@gwdg.de)
%
%%%%%%%%%%%%%%%%%%%%%%%%%%%%%%%%%%%%%%%%%
\documentclass[compress,aspectratio=169]{beamer}

% make sure the theme and config files are on this path
\usepackage[GI]{assets/beamerConfig}

\addbibresource{ref.bib}
\graphicspath{{.}{assets/}}

% --- document configuration ---
\newcommand{\mytitle}{Workflow für das Beitragen von Code zu bestehenden Projekten}
% Leave empty for no subtitle
\newcommand{\mysubtitle}{Am Beispiel von GitHub und GitHub Desktop}
\newcommand{\myauthor}{Nico Schwind}
\newcommand{\myauthorurl}{}
\newcommand{\myvenue}{Informatik 2023}
% For example, use \today
\newcommand{\mydate}{\today}
% For example, Institute for Computer Science / GWDG
\newcommand{\myinstitute}{Universität Würzburg}

\configuretitlepage

\begin{document}

	\begin{frame}[plain]
		\titlepage
	\end{frame}

	\begin{frame}[t]{Table of contents}
		\tableofcontents[subsectionstyle=hide/hide]
	\end{frame}

	% --- slides begin ---

	\section{Grundlagen}

	\begin{frame}{Möglichkeiten finden, um zu Open-Source-Projekten beizutragen}
		\begin{itemize}
			\item Nutzen sie github.com/topics/<topic>, um nach Themen zu suchen, die sie interessieren
            \item good-first-issue-Tag innerhalb eines Repositorys
            \item personalisierte Empfehlungen aufgrund von deinen bisherigen Beiträgen
            \item persönliches Dashboard
            \item GitHub Community Diskussion
		\end{itemize}
	\end{frame}
    \begin{frame}{Issue}
		\begin{itemize}
			\item Hinter Issues können Ideen, Feedback, Aufgaben oder Fehler stecken
            \item Dienen der Nachverfolgung der Arbeit am Projekt
            \item Können einzelnen Projektmitarbeitern zugewiesen werden
            \item Erster Schritt im GitHub-Flow
            \item Größere Repositorys nutzen Issuevorlagen oder Issueformulare
		\end{itemize}
	\end{frame}
 \begin{frame}{Fork}
		\begin{itemize}
            	\item Kopie des ursprünglichen Repository (Upstreamrepository)
               \item Änderungen werden nicht automatisch ins Upstreamrepository übertragen
               \item Updates im Upstreamrepository können in den Fork übernommen werden
               \item Forks verfügen über eigene Member, Branches, Tags etc.
               \item Änderungen können anschließend dem Upstreamrepository vorgeschlagen werden
		\end{itemize}
\end{frame}
\begin{frame}{Pull-Request}
\label{frame:Pull-Request}
		\begin{itemize}
            		\item informieren von Projektmitarbeitern über Änderungen die in einem Branch (Fork) gemacht wurden
              \item alle Änderungen zum ursprünglichen Branch (Fork) werden übersichtlich dargestellt
              \item Änderungen werden überprüft, kommentiert, diskutiert etc.
              \item Änderungen können nach Prüfung zusammengefügt werden
              \item Repositorys können Pull-Request Vorlagen haben
              \item Manche Repositorys haben Pull-Request Vorgaben in Form einer \glqq Contributing.md\grqq{} Datei, beispielsweise verlinken von dem gelösten Issue
		\end{itemize}
\end{frame}
\section{Schritt-für-Schritt-Anleitung}
\sectionIntroHidden
\begin{frame}{Nach Thema Suchen}
		\begin{columns}[T]
		\begin{column}{0.5\textwidth}
			\begin{itemize}
				\begin{onlyenv}<1->
					\item \href{https://github.com/topics/}{https://github.com/topics/}
				\end{onlyenv}
				\begin{onlyenv}<2->
					\item \href{https://github.com/topics/ai}{https://github.com/topics/ai}
				\end{onlyenv}
				\begin{onlyenv}<3->
					\item \href{https://github.com/topics/android}{https://github.com/topics/android}
				\end{onlyenv}
			\end{itemize}
		\end{column}
		\begin{column}{0.4\textwidth}
			\begin{tikzpicture}
				\begin{onlyenv}<1->
					\node at (0,0) (img1)	{\cutpic{0.3cm}{0.9\textwidth}{Pictures/Topic}};
				\end{onlyenv}
				\begin{onlyenv}<2->
					\node at (0,0) (img2)	{\cutpic{0.3cm}{0.9\textwidth}{Pictures/TOPIC AI}};
				\end{onlyenv}
				\begin{onlyenv}<3->
					\node at (0,0) (img3)	{\cutpic{0.3cm}{0.9\textwidth}{Pictures/Topic Android}};
				\end{onlyenv}
			\end{tikzpicture}
		\end{column}
	\end{columns}
	\source{\url{www.github.com}}
\end{frame}

\begin{frame}{Issue erstellen}
\begin{itemize} 
   \begin{onlyenv}<1>
       	    \item Besuchen sie \href{https://github.com/octocat/Spoon-Knife}{https://github.com/octocat/Spoon-Knife} und wechseln auf den Issue Tab 
   \end{onlyenv}
   \begin{onlyenv}<2>
       	    \item Auf New Issue klicken
   \end{onlyenv}
   \begin{onlyenv}<3>
       	    \item Eintragungen machen
   \end{onlyenv}
		\end{itemize}
  
			\begin{tikzpicture}
				\begin{onlyenv}<-1>
					\node at (0,7) (img1)	{\cutpic{0.3cm}{0.7\textwidth}{Pictures/IssueTAb}};
				\end{onlyenv}
				\begin{onlyenv}<2>
					\node at (0,7) (img2)	{\cutpic{0.3cm}{\textwidth}{Pictures/NewIssue}};
				\end{onlyenv}
				\begin{onlyenv}<3>
					\node at (0,7) (img3)	{\cutpic{0.3cm}{0.9\textwidth}{Pictures/Issue}};
				\end{onlyenv}
			\end{tikzpicture}
	\source{\url{www.github.com}}
\end{frame}
\begin{frame}{Repository in GitHub forken}	
\begin{itemize} 
   \begin{onlyenv}<1>
       	    \item Besuchen sie \href{https://github.com/octocat/Spoon-Knife}{https://github.com/octocat/Spoon-Knife} und klicken auf Fork 
   \end{onlyenv}
   \begin{onlyenv}<2>
       	    \item Anschließend auf \glqq Create fork\grqq{}
   \end{onlyenv}
		\end{itemize}
  
			\begin{tikzpicture}
				\begin{onlyenv}<-1>
					\node at (0,5) (img1)	{\cutpic{0.3cm}{0.9\textwidth}{Pictures/ForkSpoon}};
				\end{onlyenv}
				\begin{onlyenv}<2>
					\node at (0,5) (img2)	{\cutpic{0.3cm}{0.56\textwidth}{Pictures/CreateFork}};
				\end{onlyenv}
				
			\end{tikzpicture}
	\source{\url{www.github.com}}
\end{frame}
\begin{frame}{Repository in GitHub Desktop clonen}	
\begin{itemize} 
   \begin{onlyenv}<1>
       	    \item In dem Fork Repository auf \glqq Code \grqq{} klicken
   \end{onlyenv}
   \begin{onlyenv}<2>
       	    \item Anschließend die URL kopieren und zu GitHub Desktop wechseln
   \end{onlyenv}
   \begin{onlyenv}<3>
       	    \item In GitHub Desktop auf \glqq File \grqq{} und \glqq Clone repository\grqq{}
   \end{onlyenv}
   \begin{onlyenv}<4>
       	    \item Auf den Tab \glqq URL \grqq{} wechseln und die eben kopierte URL einfügen und auf \glqq Clone \grqq{} drücken
   \end{onlyenv}
   \begin{onlyenv}<5>
       	    \item Auswählen wie man den Fork benutzen möchte. In unserem Fall einfach auf \glqq Continue\grqq{}
   \end{onlyenv}
		\end{itemize}
  
			\begin{tikzpicture}
				\begin{onlyenv}<-1>
					\node at (0,7) (img1)	{\cutpic{0.3cm}{0.9\textwidth}{Pictures/ForkCode}};
				\end{onlyenv}
				\begin{onlyenv}<2>
					\node at (0,7) (img2)	{\cutpic{0.3cm}{0.5\textwidth}{Pictures/Copy URL}};
				\end{onlyenv}
				\begin{onlyenv}<3>
					\node at (0,7) (img2)	{\cutpic{0.3cm}{0.5\textwidth}{Pictures/GHDesktopClone}};
				\end{onlyenv}
                \begin{onlyenv}<4>
					\node at (0,7) (img2)	{\cutpic{0.3cm}{0.6\textwidth}{Pictures/GHDesktopPasteURL}};
				\end{onlyenv}
                \begin{onlyenv}<5>
					\node at (0,7) (img2)	{\cutpic{0.3cm}{0.45\textwidth}{Pictures/UseFork}};
				\end{onlyenv}
			\end{tikzpicture}
	\source{\url{www.github.com}}
\end{frame}
\begin{frame}{Ändern von Code}
\begin{columns}
    
\begin{column}{0.4\textwidth}
		\begin{itemize} 
   \begin{onlyenv}<1>
       	    \item In dem gerade geklonten Repository auf \glqq Show in Explorer\grqq{}
   \end{onlyenv}
   \begin{onlyenv}<2>
       	    \item index.html mit einem Editor öffnen
   \end{onlyenv}
   \begin{onlyenv}<3>
       	    \item Änderungen machen bspw. Name einfügen
   \end{onlyenv}
   \begin{onlyenv}<4>
       	    \item Zurück zu GitHub Desktop wechseln und unten Links einen passenden Commit-Kommentar einfügen und anschließend auf \glqq Commit to main\grqq{}
   \end{onlyenv}
   \begin{onlyenv}<5>
       	    \item Den gerade erstellten \glqq Commit\grqq{} durch \glqq Push origin\grqq{} in das Upstream-Repository übertragen
   \end{onlyenv}
		\end{itemize}
  \end{column}
  \begin{column}{0.7\textwidth}
			\begin{tikzpicture}
				\begin{onlyenv}<-1>
					\node at (0,7) (img1)	{\cutpic{0.3cm}{0.9\textwidth}{Pictures/Show in Explorer}};
				\end{onlyenv}
				\begin{onlyenv}<2>
					\node at (0,7) (img2)	{\cutpic{0.3cm}{0.9\textwidth}{Pictures/index}};
				\end{onlyenv}
				\begin{onlyenv}<3>
					\node at (0,7) (img2)	{\cutpic{0.3cm}{0.8\textwidth}{Pictures/index ändern}};
				\end{onlyenv}
                \begin{onlyenv}<4>
					\node at (0,7) (img2)	{\cutpic{0.3cm}{0.9\textwidth}{Pictures/Commit}};
				\end{onlyenv}
                \begin{onlyenv}<5>
					\node at (0,7) (img2)	{\cutpic{0.3cm}{0.8\textwidth}{Pictures/Push}};
				\end{onlyenv}
			\end{tikzpicture}
   \end{column}
	\source{\url{www.github.com}}
 \end{columns}
\end{frame}
\begin{frame}{Pull-Request}
\label{pg:lastpage} % Label on last frame to get the page number for footer
\begin{columns}
    
\begin{column}{0.4 \textwidth}
    

		\begin{itemize} 
   \begin{onlyenv}<1>
       	    \item In GitHub Desktop unter \glqq Branch\grqq{} auf \glqq Create Pull Request\grqq{}
   \end{onlyenv}
   \begin{onlyenv}<2>
       	    \item Dem Pull Request einen aussagekräftigen Titel sowie Beschreibung geben. Mit \# kann ein Issue verlinkt werden
   \end{onlyenv}
   
		\end{itemize}
\end{column}
\begin{column}{0.7\textwidth}
			\begin{tikzpicture}
				\begin{onlyenv}<1>
					\node at (0,7) (img1)	{\cutpic{0.3cm}{0.8\textwidth}{Pictures/CreatePullRequest}};
				\end{onlyenv}
				\begin{onlyenv}<2>
					\node at (0,7) (img2)	{\cutpic{0.3cm}{0.85\textwidth}{Pictures/PullRequest}};
				\end{onlyenv}
			\end{tikzpicture}
   \end{column}
   \end{columns}
	\source{\url{www.github.com}}
\end{frame}
%\begin{frame}{Dokumentation/Wiki}
%\
	%	\begin{itemize}
     %       \item  	
		%\end{itemize}
%\end{frame}

	

	%\begin{frame}{References}
		% References slide in appendix
		%\renewcommand*{\bibfont}{\normalfont\scriptsize}
		%\printbibliography[heading=none]
	%\end{frame}

\end{document}

%%%%%%%%%%%%%%%%%%%%%%%%%%%%%%%%%%%%%%%%%
% Exercise Template
% LaTeX Template
% Version 1.1 (2023-06-30)
%
% This template was adapted by:
% Jonathan Decker (jonathan.decker@uni-goettingen.de)
% From a template made by:
% Julian Kunkel (julian.kunkel@gwdg.de)
%
%%%%%%%%%%%%%%%%%%%%%%%%%%%%%%%%%%%%%%%%%
% This file provides the generic header that sets up the page formatting and
% common custom macro and environment definitions that can be used for an exercise or tutorial sheet in a university course.
%
% You can consult the `exercise.sty` file for a full list of available macros.

\documentclass[paper=a4]{scrartcl}

\newcommand{\theauthor}{Marc-Philipp Knechtle}
\newcommand{\thecourse}{Open Source Workshop} % Course long name
\newcommand{\thecourseabbr}{AK-OSS Workshop} % Course abbreviation
\newcommand{\thetype}{Exercise} % e.g. Exercise, Tutorial
\newcommand{\theorganization}{Informatik 2023} % e.g. University Göttingen
\newcommand{\theinstitution}{Arbeitskreis Open Source Software} % e.g. Institute of Computer Science
\newcommand{\theyear}{2023} % The year
\newcommand{\theexercisenumber}{1} % number of the exercise
\newcommand{\thedate}{September 27, 2023} % e.g. March 28, 2022
\newcommand{\theterm}{AK-OSS} % SoSe or WiSe 

\usepackage{exercise} % Custom header and macros

\begin{document}
\date{\thedate}
\exercise{\theexercisenumber}

\parskip 8pt
\makesheetheader

\section*{Exercise Introduction}
This sheet covers additional exercise material for the presentation about publishing your own open source code.

\tableofcontents

\bigskip

\task{Hide your secrets}{10}

	\subsection*{Task Preparation}
		\begin{enumerate}
			\item clone \href{https://github.com/marc-philipp-knechtle/2023-informatik-workshop-publish-code}{2023-informatik-workshop-publish-code}
			\item create a venv inside this repository
			\item activate the venv
			\item install all dependencies
			\item download the \texttt{bfg} repo cleaner \href{https://rtyley.github.io/bfg-repo-cleaner/}{here}
		\end{enumerate}

	\subsection*{Task part 2}
	\begin{enumerate}
		\item \texttt{echo "Hello World"}
	\end{enumerate}

	\begin{hints} % Give tips or hints that are helpful
  		\item 
	\end{hints}

	% \begin{submissions}{submission-folder} % The expected form of the results of completing the task
	%	\submission{}findings.txt}{Write about X (at least 100 words).}
	%\end{submissions}

	\begin{literature} % Further Reading
  		\item \href{https://docs.github.com/en/authentication/keeping-your-account-and-data-secure/removing-sensitive-data-from-a-repository}{removing sensitive data from a repository}
	\end{literature}


	\begin{tcolorbox}[title=https-server.py] % Include code blocks like this
		\inputminted{Python}{http-server.py}
		\embedfile{http-server.py} % Use embedfile package to enable download of the code
	\end{tcolorbox}
	\marginpar{\attachfile[appearance=false,icon=Paperclip,mimetype=text/plain]{http-server.py}} % Just in case someone's PDF reader has better support for attached files

% \hardtask{Hard Task} % More difficult alternative to above task, uses the same time and number as the respective task

% \extraTask{Extra Task}{5} % Additional tasks that are optional

\end{document}
